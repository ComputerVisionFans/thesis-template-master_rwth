\documentclass[a4paper, 13pt]{scrartcl}
\usepackage[left=2cm, top=2.8cm,bottom=0cm,head=1cm, foot=0cm,headsep=1.2cm]{geometry}
\usepackage{xcolor}
\usepackage{graphicx}
\usepackage{tabularx}

% variables
\usepackage{titling}
\makeatletter
\newcommand{\course}{Master Informatik}
\newcommand{\degree}{Master Informatik}
\newcommand{\matrnumber}{384394}
\author{Qiang, Li}
\newcommand{\telephone}{}
\date{2020-08-15}
\makeatother

% language
\usepackage[ngerman]{babel}
\usepackage[utf8]{inputenc}
% \usepackage[iso]{isodate}

\setlength{\parindent}{0pt}
\usepackage{fancyhdr}
\fancyhf{}
\renewcommand{\headrulewidth}{0pt}
\fancyhead[RO]{\includegraphics[height=1.5cm]{images/logo/rwth.pdf}}
\fancyhead[LO]{\large Zentrales Prüfungsamt}

\begin{document}
	\sffamily
	\pagestyle{fancy}

	\begin{center}
		\LARGE\bfseries
			\underline{Cell Morphology Based Diagnosis of Cancer}
			\underline{Using Convolutional Neural Networks}
	\end{center}

	\vspace{1cm}

	\begin{tabularx}{\textwidth}{@{}l *1{>{\centering\arraybackslash}X}@{}}
		Studiengang: & {\bfseries\course}\\
		\cline{2-2}
		& \\[0.1cm]
		Abschluss: & {\bfseries\degree}\\
		\cline{2-2}
	\end{tabularx}

	\vspace{0.5cm}

	\begin{minipage}[t]{0.4\textwidth}
		\begin{tabularx}{0.8\textwidth}{*1{>{\centering\arraybackslash}X}@{}}
			\bfseries\matrnumber\\
			\cline{1-1}
			\multicolumn{1}{l}{Matrikelnummer}
		\end{tabularx}
	\end{minipage}
	\begin{minipage}[t]{0.59\textwidth}
		\begin{tabularx}{\textwidth}{*1{>{\centering\arraybackslash}X}@{}}
			\bfseries\theauthor\\
			\cline{1-1}
			\multicolumn{1}{l}{Name, Vorname}
		\end{tabularx}
	\end{minipage}

	\vspace{0.5cm}

	\begin{minipage}[t]{0.4\textwidth}
		\begin{tabularx}{0.8\textwidth}{*1{>{\centering\arraybackslash}X}@{}}
			\telephone\\
			\cline{1-1}
			\multicolumn{1}{l}{Telefonnr. für Rückfragen}
		\end{tabularx}
	\end{minipage}

	\vspace{0.5cm}

	\begin{tabularx}{\textwidth}{@{}l *1{>{\centering\arraybackslash}X}@{}}
		Bezeichnung der Prüfungsarbeit: & {\bfseries Abschlussarbeit}\\
		\cline{2-2}
	\end{tabularx}
	{\small (z.B. Hausarbeit, Projektarbeit, Abschlussarbeit)}

	\vspace{0.5cm}

	Name der Institution, deren Logo verwendet werden soll:\\[0.2cm]
	\begin{tabularx}{\textwidth}{*1{>{\centering\arraybackslash}X}}
		{\bfseries Visual Computing Institute}\\
		\cline{1-1}
	\end{tabularx}

	\vspace{0.2cm}

	Mir ist bekannt, dass:
	\begin{itemize}
		\item[-] ich das kombinierte Logo nur verwenden darf, wenn es von der gewünschten Institution für die Nutzung auf Prüfungsarbeiten freigegeben ist
		\item[-] ich das kombinierte Logo selbständig und zeitnah bei der gewünschten Institution anfordern muss
		\item[-] das Logo nicht verändert, beschnitten oder aus der Logosystematik losgelöst werden darf
		\item[-] das RWTH Aachen University-Logo eine eingetragene und geschützte Wortmarke ist und ein Verstoß gegen die Vorgaben zur Verwendung des Logos Rechtsfolgen hat
		\item[-] das kombinierte Logo nur für die hier angegebene Arbeit verwendet werden darf
		\item[-] das kombinierte Logo nur mit dem Zusatz \glqq Diese Arbeit wurde vorgelegt am [Name Institution]\grqq angebracht werden darf
		\item[-] das kombinierte Logo nur rechts oben auf dem Deckblatt angebracht und in keiner Art verändert werden darf
	\end{itemize}

	Diese und weitere Informationen (inkl. Anwendungsbeispiel) kann ich auf der \textcolor{blue}{\underline{Webseite des ZPA}} jederzeit nachlesen.

	\vspace{1cm}

	\begin{minipage}[t]{0.4\textwidth}
		\begin{tabularx}{0.8\textwidth}{*1{>{\centering\arraybackslash}X}@{}}
			\bfseries\thedate\\
			\cline{1-1}
			\multicolumn{1}{l}{Datum}
		\end{tabularx}
	\end{minipage}
	\begin{minipage}[t]{0.59\textwidth}
		\begin{tabularx}{\textwidth}{*1{>{\centering\arraybackslash}X}@{}}
			\\
			\cline{1-1}
			\multicolumn{1}{l}{Unterschrift}
		\end{tabularx}
	\end{minipage}

	\vspace{0.5cm}

	\begin{center}
		\scriptsize Diese Bescheinigung ist zu Beginn der schriftlichen Prüfungsarbeit beim ZPA einzureichen.
	\end{center}
\end{document}

